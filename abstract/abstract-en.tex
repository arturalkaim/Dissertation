%!TEX root = ../dissertation.tex

\begin{otherlanguage}{english}
\begin{abstract}
% Set the page style to show the page number
\thispagestyle{plain}
\abstractEnglishPageNumber
Architects and designers, to create their models, largely use \gls{CAD} tools. This tools are very powerful for modeling and manipulation of this models, but they are mostly geared for manual use. Unfortunately, the manual production of large amounts geometry is very time consuming.

Procedural generation of these forms is one of the approaches which considerably speeds up this process. This approach consists in an algorithmic construction of forms and allows the quick creation of massive amounts of geometry. As most 3D modeling tools were not made specifically for this type of use, favoring instead manual use, they do not have the performance necessary for a smooth use. 

This work proposes solutions to this performance problem, through the use of different techniques that accelerate the production and visualization of large volumes of geometry. It is a library that implements various techniques and provides a 3D Modeling API with a Racket interface that accesses this library.
% Keywords
\begin{flushleft}

\keywords{3D modeling, OpenGL, Generative Design, Shaders, Level of Detail}

\end{flushleft}

\end{abstract}
\end{otherlanguage}
