%!TEX root = ../dissertation.tex

\begin{otherlanguage}{portuguese}
\begin{abstract}
\abstractPortuguesePageNumber
Arquitetos e designers, para criar seus modelos, usam  ferramentas de Desenho Assistido por Computador (DAC). Esta ferramentas são muito poderosas para modelação e manipulação dos modelos, mas elas são em sua maioria desenvolvidas para o uso manual. Infelizmente, a produção de grandes quantidades manual de geometria é muito tempo consumindo.

Geração processual destas formas é uma das abordagens que acelera consideravelmente o processo. Esta abordagem consiste na construção de algoritmos de formas e permite a criação rápida de grandes quantidades de geometria. Como a maioria das ferramentas de modelagem 3D não foram feitas especificamente para este tipo de utilização, favorecendo o uso em vez manuais, eles não têm o desempenho necessário para um uso suave.

Este trabalho propõe soluções para este problema de desempenho, através da utilização de diferentes técnicas que aceleram a produção e visualização de grandes volumes de geometria. É uma biblioteca que implementa várias técnicas e fornece uma API Modelagem 3D com uma interface Racket construído que acessa esta biblioteca.

% Keywords
\begin{flushleft}

\palavrasChave{as tuas palavras chave}

\end{flushleft}

\end{abstract}
\end{otherlanguage}
